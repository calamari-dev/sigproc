\documentclass[../../main]{subfiles}

\begin{document}
\chapter{演習問題の略解}
\label{chapter:answer_of_exercises}
\small

\section*{第2章}

\begin{enumerate}
  \item \(\mat{A}=\zmat\)のときは\(\mat{\Sigma}=\zmat\)とすればよい.
    \(\mat{A}\neq\zmat\),\(n<p\)のとき,\(\htrps{\mat{A}}\)の特異値分解を\(\htrps{\mat{A}}=\mat{V}\mat{\Sigma}\htrps{\mat{U}}\)とおくと\(\mat{A}=\mat{U}\htrps{\mat{\Sigma}}\htrps{\mat{V}}\)である.
    \(\htrps{\mat{\Sigma}}\)をあらためて\(\mat{\Sigma}\)とすれば\(\mat{A}=\mat{U}\mat{\Sigma}\htrps{\mat{V}}\)であり,これは\(\mat{A}\)の特異値分解である.
  \item 左側を示す.\cref{xr-proposition:pseudoinverse_characterization}より,任意の\(\vect{n}\in\nulsp\mat{A}\)に対して
    \(
      \innerp{(\imat-\pinv{\mat{A}}\mat{A})\vect{x}-\vect{x}}{\vect{n}} = -\innerp{\vect{x}}{\htrps{(\pinv{\mat{A}}\mat{A})}\vect{n}}
      = -\innerp{\vect{x}}{\pinv{\mat{A}}\mat{A}\vect{n}}
      = 0
    \)
    である.よって\(\proj_{\nulsp\mat{A}}\vect{x}=(\imat-\pinv{\mat{A}}\mat{A})\vect{x}\)である.
    同様に,任意の\(\mat{A}\vect{w}\in\colsp\mat{A}\)に対して
    \(
      \innerp{\mat{A}\pinv{\mat{A}}\vect{y}-\vect{y}}{\mat{A}\vect{w}} = \innerp{\htrps{\mat{A}}\htrps{(\mat{A}\pinv{\mat{A}})}\vect{y}-\htrps{\mat{A}}\vect{y}}{\vect{w}}
      = \innerp{(\htrps{(\mat{A}\pinv{\mat{A}}\mat{A})}-\htrps{\mat{A}})\vect{y}}{\vect{w}}
      = 0
    \)
    である.したがって\(\proj_{\colsp\mat{A}}\vect{y}=\mat{A}\pinv{\mat{A}}\vect{y}\)である.
  \item \(\mat{\Lambda}=\diag(\lambda_1,\dots,\lambda_n)\)とおくと,\(\mat{A}^k=(\mat{U}\mat{\Lambda}\htrps{\mat{U}})^k=\mat{U}\mat{\Lambda}^k\htrps{\mat{U}}\)より
    \[
      \exp\mat{A} = \sum\frac{\mat{A}^k}{k!}
      = \sum\frac{\mat{U}\mat{\Lambda}^k\htrps{\mat{U}}}{k!}
      = \mat{U}\pqty*{\sum\frac{\diag(\lambda_1^k,\dots,\lambda_n^k)}{k!}}\htrps{\mat{U}}
    \]
    であり,右辺は\(\mat{U}\diag(\napr^{\lambda_1},\dots,\napr^{\lambda_n})\htrps{\mat{U}}\)に等しい.
\end{enumerate}

\end{document}
