\documentclass[../../main]{subfiles}

\begin{document}
\chapter{演習問題の略解}
\label{chapter:answer_of_exercises}
\small

\section*{第2章}

\begin{enumerate}
  \item \(\mat{A}=\zmat\)のときは\(\mat{\Sigma}=\zmat\)とすればよい.
    \(\mat{A}\neq\zmat\),\(n<p\)のとき,\(\htrps{\mat{A}}\)の特異値分解を\(\htrps{\mat{A}}=\mat{V}\mat{\Sigma}\htrps{\mat{U}}\)とおくと\(\mat{A}=\mat{U}\htrps{\mat{\Sigma}}\htrps{\mat{V}}\)である.
    \(\htrps{\mat{\Sigma}}\)をあらためて\(\mat{\Sigma}\)とすれば\(\mat{A}=\mat{U}\mat{\Sigma}\htrps{\mat{V}}\)であり,これは\(\mat{A}\)の特異値分解である.
  \item 左側を示す.\cref{xr-proposition:pseudoinverse_characterization}より,任意の\(\vect{n}\in\nulsp\mat{A}\)に対して
    \(\innerp{(\imat-\pinv{\mat{A}}\mat{A})\vect{x}-\vect{x}}{\vect{n}} = -\innerp{\vect{x}}{\htrps{(\pinv{\mat{A}}\mat{A})}\vect{n}}=-\innerp{\vect{x}}{\pinv{\mat{A}}\mat{A}\vect{n}}=0\)
    だから\(\proj_{\nulsp\mat{A}}\vect{x}=(\imat-\pinv{\mat{A}}\mat{A})\vect{x}\)である.同様に,任意の\(\mat{A}\vect{w}\in\colsp\mat{A}\)に対して
    \(\innerp{\mat{A}\pinv{\mat{A}}\vect{y}-\vect{y}}{\mat{A}\vect{w}}=\innerp{\htrps{\mat{A}}\htrps{(\mat{A}\pinv{\mat{A}})}\vect{y}-\htrps{\mat{A}}\vect{y}}{\vect{w}}=\innerp{(\htrps{(\mat{A}\pinv{\mat{A}}\mat{A})}-\htrps{\mat{A}})\vect{y}}{\vect{w}}=0\)
    だから\(\proj_{\colsp\mat{A}}\vect{y}=\mat{A}\pinv{\mat{A}}\vect{y}\)である.
  \item \(\mat{\Lambda}=\diag(\lambda_1,\dots,\lambda_n)\)とおくと,\(\mat{A}^k=(\mat{U}\mat{\Lambda}\htrps{\mat{U}})^k=\mat{U}\mat{\Lambda}^k\htrps{\mat{U}}\)より
    \[
      \exp\mat{A} = \sum\frac{\mat{A}^k}{k!}
      = \sum\frac{\mat{U}\mat{\Lambda}^k\htrps{\mat{U}}}{k!}
      = \mat{U}\pqty*{\sum\frac{\diag(\lambda_1^k,\dots,\lambda_n^k)}{k!}}\htrps{\mat{U}}
    \]
    であり,右辺は\(\mat{U}\diag(\napr^{\lambda_1},\dots,\napr^{\lambda_n})\htrps{\mat{U}}\)に等しい.
  \item \cref{xr-proposition:unitary_matrix_characterization}と\cref{xr-proposition:dft_onb}からしたがう.
  \item \(\vect{\phi}_0=\trps{\rowvect{0 & 1}}\),\(\vect{\phi}_1=\trps{\rowvect{-\sqrt{3} & -1}}/2\),\(\vect{\phi}_2=\trps{\rowvect{\sqrt{3} & -1}}/2\)より
    \[
      \sum_{k=0}^2\abs{\innerp{\vect{x}}{\vect{\phi}_k}}^2 = v^2+\pqty*{-\frac{\sqrt{3}u}{2}-\frac{v}{2}}^2+\pqty*{\frac{\sqrt{3}u}{2}-\frac{v}{2}}^2\quad(\vect{x}=\trps{\matrice{u & v}})
    \]
    である.右辺を整理すると\((3/2)(u^2+v^2)=(3/2)\vnorm{\vect{x}}^2\)となる.
  \item \(x\in\cycles{N}\)を任意にとる.\((\dgt{w}x)[k,j]=(\stft{w}x)[k,jN]=\sqrt{N}\dft{N}(x\cdot\translate{jN}\conj{w})[k]\),\(\translate{jN}w=w\)なので\((\dgt{w}x)[k,j]=\sqrt{N}\dft{N}(x\cdot\conj{w})[k]\)である.
    仮に\ltjkenten{ある}\(m\)で\(w[m]=0\)だったとすると,\(x=\translate{m}\comb\)のとき
    \[
      (\dgt{w}x)[k,j] = \sum_{n=0}^{N-1}\comb[n-m]\conj*{w[n]}\napr^{-2\krez\iuni kn/N}
      = \conj*{w[m]}\napr^{-2\krez\iuni km/N}
      = 0
    \]
    となる.よって\(\translate{m}\comb\in\ker\dgt{w}\)だから,\(\dgt{w}\)は単射ではない.一方,\cref{xr-corollary:frame_bijection}より任意のフレームについて分解作用素は単射である.したがって,このとき\(\gaborsys{w}{K}{N}\)はフレームではない.
\end{enumerate}

\end{document}
