\documentclass[../../main]{subfiles}

\begin{document}
\chapter{記号について}
\thispagestyle{empty}
書籍ごとに異なることが多い記号について,記号と定義の組を示す.
表にない記号については,巻末の索引を参照のこと.

\vspace*{\fill}
\begin{table*}
  \centering
  \begin{tabular}{c|l} \hline
    記号 & \multicolumn{1}{c}{定義} \\ \hline
    \(\holder\) & プレースホルダ \\
    \(\numset{N}\) & \(\Set{1,2,\dotsc}\) \\
    \(\numset{Z}\) & \(\Set{\dotsc,-2,-1,0,1,2,\dotsc}\) \\
    \(\numset{K}\) & 実数体\(\numset{R}\)か複素数体\(\numset{C}\) \\
    \(\scomp{\pholder}\) & 補集合 \\
    \(\clsr\pholder\) & 閉包 \\
    \(\kdelta{i}{j}\) & クロネッカーのデルタ \\
    \(\innerp{\holder}{\holder}\) & 内積 \\
    \(\vnorm{\holder}\) & ノルム \\
    \(\imat\) & 単位行列 \\
    \(\zmat\) & 零行列 \\
    \(\trps{\pholder}\) & 転置 \\
    \(\htrps{\pholder}\) & エルミート転置 \\
    \(\fnorm{\holder}\) & フロベニウスノルム \\
    \(\dft{N}\) & 離散フーリエ変換 \\
    \(\dtft\) & 離散時間フーリエ変換 \\
    \(\fseries\) & フーリエ係数列 \\
    \(\ftrans\) & フーリエ変換 \\ \hline
  \end{tabular}
\end{table*}
\vspace*{\fill}

\end{document}
