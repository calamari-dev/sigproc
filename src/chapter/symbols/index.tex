\documentclass[../../main]{subfiles}

\begin{document}
\chapter{記号について}
\thispagestyle{empty}
書籍ごとに異なることが多い記号について,記号と定義の組を示す.
表にない記号については,巻末の索引を参照のこと.

\vspace*{\fill}
\begin{table*}
  \centering
  \begin{tabular}{c|l} \hline
    記号 & \multicolumn{1}{c}{定義} \\ \hline
    \(\numset{N}\) & 自然数の全体集合\(\Set{1,2,\dotsc}\) \\
    \(\numset{Z}\) & 整数の全体集合\(\Set{\dotsc,-2,-1,0,1,2,\dotsc}\) \\
    \(\numset{K}\) & 実数の全体集合\(\numset{R}\)か複素数の全体集合\(\numset{C}\) \\
    \(\scomp{S}\) & 集合\(S\)の補集合 \\
    \(\clsr S\) & 集合\(S\)の閉包 \\
    \(\kdelta{i}{j}\) & クロネッカーのデルタ \\
    \(\innerp{\vect{u}}{\vect{v}}\) & ベクトル\(\vect{u}\),\(\vect{v}\)の内積 \\
    \(\vnorm{\vect{v}}\) & ベクトル\(\vect{v}\)のノルム \\
    \(\imat\) & 単位行列 \\
    \(\zmat\) & 零行列 \\
    \(\trps{\mat{A}}\) & 行列\(\mat{A}\)の転置行列 \\
    \(\htrps{\mat{A}}\) & 行列\(\mat{A}\)のエルミート転置 \\
    \(\fnorm{\mat{A}}\) & 行列\(\mat{A}\)のフロベニウスノルム \\
    \(\dft{N}\) & 離散フーリエ変換 \\
    \(\dtft\) & 離散時間フーリエ変換 \\
    \(\fseries\) & フーリエ係数列 \\
    \(\ftrans\) & フーリエ変換 \\ \hline
  \end{tabular}
\end{table*}
\vspace*{\fill}

\end{document}
