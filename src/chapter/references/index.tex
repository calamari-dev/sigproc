\documentclass[../../main]{subfiles}

\begin{document}
\phantomsection
\addcontentsline{toc}{chapter}{\refname}
\begin{thebibliography}{99}
  \bibitem{arai2010}
    新井仁之. ウェーブレット. 共立出版, 2010, 463p., (共立叢書 現代数学の潮流, 10).

  \bibitem{canon}
    カノン. “波音リツ音源配布所”. カノンの落ちる城. \url{http://www.canon-voice.com/index.html}, (参照 2022-11-25).

  \bibitem{casazza2013}
    Casazza, Peter G. et al. \textit{Finite Frames: Theory and Applications}. Birkhäuser Boston, 2013, 485p., (online), available from SpringerLink, (accessed 2022-08-09).

  \bibitem{funaki2022}
    舟木直久. 確率論. 朝倉書店, 2022, 261p., (講座 数学の考え方, 20).

  \bibitem{kashino}
    柏野牧夫. “ピッチと基本周波数はどう違うのですか。”. 日本音響学会. \url{https://acoustics.jp/qanda/answer/101.html}, (参照 2022-12-18).

  \bibitem{kuroda2021}
    黒田成俊. 関数解析. 共立出版, 2021, 339p., (共立数学講座, 15).

  \bibitem{luenburger1969}
    Luenberger, David G. \textit{Optimization by Vector Space Methods}. Wiley, 1969, 326p.

  \bibitem{matsuzaka2018}
    松坂和夫. 集合・位相入門. 岩波書店, 2018, 329p., (松坂和夫 数学入門シリーズ, 1).

  \bibitem{morise2018}
    森勢将雅. 日本音響学会編. 音声分析合成. コロナ社, 2018, 272p., (音響テクノロジーシリーズ, 22).

  \bibitem{saito2020}
    齋藤正彦. 線型代数入門. 東京大学出版会, 2020, 274p., (基礎数学, 1).

  \bibitem{sugiura2018}
    杉浦光夫. 解析入門I. 東京大学出版会, 2018, 442p., (基礎数学, 2).

  \bibitem{yanai2011}
    Yanai, Haruo. et al. \textit{Projection Matrices, Generalized Inverse Matrices, and Singular Value Decomposition}. Springer New York, 2011, 243p., (online), available from SpringerLink, (accessed 2022-08-22).

  \bibitem{yatabe2019}
    矢田部浩平ほか. 小特集, 位相情報を考慮した音声音響信号処理: 位相変換による複素スペクトログラムの表現. 日本音響学会誌. 2019, vol. 75, no. 3, p. 147-155. \url{https://www.jstage.jst.go.jp/article/jasj/75/3/75_147/_article/-char/ja/}, (参照 2023-03-15).

  \bibitem{yatabe2021}
    矢田部浩平. 第三回: 短時間フーリエ変換. 日本音響学会誌. 2021, vol. 77, no. 6, p. 396-403. \url{https://www.jstage.jst.go.jp/article/jasj/77/6/77_396/_article/-char/ja/}, (参照 2023-03-17).

  \bibitem{yukie2019}
    雪江明彦. 環と体とガロア理論. 日本評論社, 2019, 300p., (代数学, 2).
\end{thebibliography}
\end{document}
