\documentclass[../../main]{subfiles}

\begin{document}
\chapter{測度空間}

\section{イントロダクション}

\section{測度論の基本概念}

\subsection{σ‐加法族}

\begin{definition}{σ‐加法族}{sigma_algebra}\index{しぐまかほうぞく@σ‐加法族}
  \(\Omega\)を集合,\(\sigmaalg{F}\)を\(\powerset{\Omega}\)の部分集合とする.
  \(\sigmaalg{F}\)が\(\Omega\)上の\termdef{σ‐加法族}(σ‐algebra)であるとは,\(\sigmaalg{F}\)が以下の条件を満たすことをいう.
  \begin{enumerate}
    \item \(\Omega\in\sigmaalg{F}\)である.
    \item 任意の\(A\in\sigmaalg{F}\)に対して\(\scomp{A}=\Omega\setminus A\in\sigmaalg{F}\)である.
    \item 任意の\(A_1,A_2,\dotsc\in\sigmaalg{F}\)に対して\(\bigcup_{n\in\numset{N}}A_n\in\sigmaalg{F}\)である.
  \end{enumerate}
\end{definition}

組\((\Omega,\sigmaalg{F})\)を\termdef{可測空間}\index{かそくくうかん@可測空間}(measurable space)という.

\begin{example}
  \(\Set{\emptyset,\Omega}\)と\(\powerset{\Omega}\)は\(\Omega\)上のσ‐加法族である.
\end{example}

\begin{example}\label{example:dice}
  \(\Omega=\Set{\dicei,\diceii,\diceiii,\diceiv,\dicev,\dicevi}\)を6元集合とし,\(O=\Set{\dicei,\diceiii,\dicev}\)とおく.
  このとき,\(\sigmaalg{G}=\Set{\emptyset,O,\scomp{O},\Omega}\)は\(\Omega\)上のσ‐加法族である.
\end{example}

\begin{definition}{生成するσ‐加法族}{}\index{せいせい@生成!しぐまかほうぞく@σ‐加法族}\index{しぐまかほうぞく@σ‐加法族!せいせいする@生成する\twoemdash}\indexsymbol{\(\sigmagen{S}\)}
  \(\Omega\)を集合,\(S\)を\(\powerset{\Omega}\)の部分集合とする.また,\(S\)を包含する\(\Omega\)上のσ‐加法族全体を\(\Sigma(S)\)とおく.このとき,集合
  \[
    \sigmagen{S} = \bigcap_{\sigmaalg{F}\in\Sigma(S)}\sigmaalg{F}
  \]
  は\(\Sigma(S)\)に属し,\(\Sigma(S)\)の任意の元は\(\sigmagen{S}\)を包含する.\(\sigmagen{S}\)を\(S\)が\termdef{生成するσ‐加法族}(generated σ‐algebra)という.
\end{definition}

\begin{example}[ボレル集合族]\index{ぼれるしゅうごうぞく@ボレル集合族}\index{BR@\(\borelset{\numset{R}}\)}
  左半開区間の全体集合\(S=\Set{\ocival{-\infty}{a}\given a\in\numset{R}}\)により生成されるσ‐加法族を\termdef{ボレル集合族}(Borel algebra)といい,\(\borelset{\numset{R}}\)と表記する.
\end{example}

\begin{note}
  \(\borelset{\numset{R}}\)は\(\numset{R}\)の開集合系が生成するσ‐加法族でもある.実はより一般に,位相空間の開集合系が生成するσ‐加法族のことをボレル集合族という.
\end{note}

\begin{example}
  \cref{example:dice}のσ‐加法族\(\sigmaalg{G}\)は\(\sigmaalg{G}=\sigmagen{\Set{O}}\)と書ける.
  実際,\(\sigmaalg{F}\)が\(\Set{O}\)を包含するσ‐加法族なら\(\scomp{O}\in\sigmaalg{F}\),\(\sigmaalg{G}\subset\sigmaalg{F}\)である.
  つまり,\(\sigmaalg{G}\)は\(\Set{O}\)を包含する最小のσ‐加法族だから,\(\sigmaalg{G}=\sigmagen{\Set{O}}\)である.
\end{example}

\subsection{ボレル測度とルベーグ測度}

\begin{definition}{拡大実数}{extended_real}\index{かくだいじっすう@拡大実数}\index{R@\(\extendedreal\)}
  \(\numset{R}\)に正負の無限大\(+\infty,-\infty\notin\numset{R}\)を加えた集合\(\extendedreal=\numset{R}\cup\Set{\pm\infty}\)を\termdef{拡大実数}(extended real number)という.
  各\(a\in\numset{R}\),\(x\in\extendedreal\)に対し,\(\extendedreal\)における演算を以下の通り定義する(複合同順).
  \begin{gather*}
    a+(\pm\infty) = \pm\infty,
    \quad a-(\pm\infty) = \mp\infty,
    \quad(\pm\infty)+(\pm\infty) = \pm\infty, \\
    (\pm\infty)-(\mp\infty) = \pm\infty,
    \quad\frac{a}{\pm\infty} = 0,
    \quad x\times(\pm\infty) = \begin{cases}\pm\infty & (x>0),\\ 0 & (x=0),\\ \mp\infty & (x<0)\end{cases}
  \end{gather*}
\end{definition}

\(\seq{A_n}_{n\in\numset{N}}\)を集合列とする.\(\seq{A_n}_{n\in\numset{N}}\)が\termdef{互いに素}\index{たがいにそ@互いに素}(disjoint)であるとは,
異なる任意の2数\(i,j\in\numset{N}\)に対して\(A_i\cap A_j=\emptyset\)であることをいう.
互いに素な集合の和であることを強調したいときは,和集合\(\bigcup_nA_n\)を\(\bigsqcup_nA_n\)とも書く.

\begin{definition}{測度}{measure}\index{そくど@測度}\index{そくど@測度}
  \(\sigmaalg{F}\)をσ‐加法族とする.写像\(\mu\colon\sigmaalg{F}\to\ccival{0}{+\infty}\)が\(\sigmaalg{F}\)上の\termdef{測度}(measure)であるとは,
  \(\mu\)が以下の条件を満たすことをいう.
  \begin{enumerate}
    \item \(\mu(\emptyset)=0\)である.
    \item \(\sigmaalg{F}\)上の集合列\(\seq{A_n}_{n\in\numset{N}}\)が互いに素であるとき,\(\mu(\bigsqcup_{n\in\numset{N}}A_n)=\sum_{n=1}^\infty\mu(A_n)\)である.
  \end{enumerate}
\end{definition}

3つ組\((\Omega,\sigmaalg{F},\mu)\)を\termdef{測度空間}\index{そくどくうかん@測度空間}(measure space)という.
特に\(\mu(\Omega)=1\)であるとき,\(\mu\)を\termdef{確率測度}\index{かくりつそくど@確率測度}\index{そくど@測度!かくりつ@確率\twoemdash}(probability measure),\((\Omega,\sigmaalg{F},\mu)\)を\termdef{確率空間}\index{かくりつくうかん@確率空間}(probability space)という.
また,確率空間のσ‐加法族\(\sigmaalg{F}\)に属する各元は\termdef{事象}\index{じしょう@事象}(event)と呼ばれる.

\begin{example}[計数測度]\index{けいすうそくど@計数測度}
  \(\sigmaalg{F}\)をσ‐加法族とする.また,各\(A\in\sigmaalg{F}\)に対して,\(A\)が有限集合のとき\(\mu(A)=\sizeof{A}\),無限集合のとき\(\mu(A)=+\infty\)とする.
  このとき\(\mu\)は\(\sigmaalg{F}\)上の測度になる.\(\mu\)を\termdef{計数測度}(counting measure)という.
\end{example}

\begin{example}[ディラック測度]\index{でぃらっくそくど@ディラック測度}
  \((\Omega,\sigmaalg{F})\)を可測空間とし,任意に\(x\in\Omega\)をとる.また,各\(A\in\sigmaalg{F}\)に対して,\(x\in A\)のとき\(\delta_x(A)=1\),\(x\notin A\)のとき\(\delta_x(A)=0\)とする.
  このとき\(\delta_x\)は\(\sigmaalg{F}\)上の測度になる.\(\delta_x\)を\termdef{ディラック測度}(Dirac measure)という.
\end{example}

\begin{example}\label{example:equally_possible}
  \(\Omega\)を\cref{example:dice}と同じにし,\(\sigmaalg{F}=\powerset{\Omega}\)とする.このとき,写像\(\prob\colon\sigmaalg{F}\to\ccival{0}{1}\),\(\prob(A)=(1/6)\sizeof{A}\)は\(\sigmaalg{F}\)上の確率測度である.
  \(\prob\)は「どの目が出るのも同様に確からしい(公平な)6面ダイスに関する確率」を表すと解釈できる.たとえば,奇数の目が出る確率は\(\prob(O)=1/2\)である.
\end{example}

\begin{definition}{有限加法族}{finite_additive_class}\index{ゆうげんかほうぞく@有限加法族}
  \(\Omega\)を集合,\(\sigmaalg{A}\)を\(\powerset{\Omega}\)の部分集合とする.
  \(\sigmaalg{A}\)が\(\Omega\)上の\termdef{有限加法族}(finitely additive class)であるとは,\(\sigmaalg{A}\)が以下の条件を満たすことをいう.
  \begin{enumerate}
    \item \(\Omega\in\sigmaalg{A}\)である.
    \item 任意の\(A\in\sigmaalg{A}\)に対して\(\scomp{A}=\Omega\setminus A\in\sigmaalg{A}\)である.
    \item 任意の\(A,B\in\sigmaalg{A}\)に対して\(A\cup B\in\sigmaalg{A}\)である.
  \end{enumerate}
\end{definition}

\begin{definition}{有限加法的測度}{finitely_additive_measure}\index{ゆうげんかほうてきそくど@有限加法的測度}\index{そくど@測度!ゆうげんかほうてき@有限加法的\twoemdash}
  \(\sigmaalg{A}\)を有限加法族とする.写像\(m\colon\sigmaalg{A}\to\ccival{0}{+\infty}\)が\(\sigmaalg{A}\)上の\termdef{有限加法的測度}(finitely additive measure)であるとは,
  \(m\)が以下の条件を満たすことをいう.
  \begin{enumerate}
    \item \(m(\emptyset)=0\)である.
    \item \(A,B\in\sigmaalg{A}\)が\(A\cap B=\emptyset\)を満たすとき,\(m(A\sqcup B)=m(A)+m(B)\)である.
  \end{enumerate}
\end{definition}

\begin{note}
  infiniteは「インフィニット」と読むが,finiteは「ファイナイト」と読む.
\end{note}

\begin{theorem}{ホップの拡張定理}{hopf_extension}\index{ほっぷのかくちょうていり@ホップの拡張定理}
  \(\sigmaalg{A}\)を集合\(\Omega\)上の有限加法族,\(m\)を\(\sigmaalg{A}\)上の有限加法的測度とする.このとき,以下の命題は同値である.
  \begin{enumerate}
    \item \(\sigmaalg{A}\)上の集合列\(\seq{A_n}_{n\in\numset{N}}\)が互いに素で\(A=\bigsqcup_{n\in\numset{N}}A_n\in\sigmaalg{A}\)を満たすとき,\(m(A)=\sum_{n=1}^\infty m(A_n)\)である.
    \item \(\sigmagen{\sigmaalg{A}}\)上の測度\(\mu\)で,任意の\(A\in\sigmaalg{A}\)に対して\(\mu(A)=m(A)\)を満たすものが存在する.
  \end{enumerate}
  さらに,\(\sigmaalg{A}\)上の集合列\(\seq{A_n}_{n\in\numset{N}}\)で\(m(A_n)<+\infty\),\(\bigcup_{n\in\numset{N}}A_n=\Omega\)を満たすものが存在するとき,\(\mu\)は一意である.
  これを\termdef{ホップの拡張定理}(Hopf extension theorem)という.
\end{theorem}

\begin{note}
  \(\sigmaalg{A}\)上の集合列\(\seq{A_n}_{n\in\numset{N}}\)で\(m(A_n)<+\infty\),\(\bigcup_{n\in\numset{N}}A_n=\Omega\)を満たすものが存在するとき,
  \(m\)は\termdef{σ‐有限}\index{しぐまゆうげん@σ‐有限}(σ‐finite)であるという.本書が扱う(有限加法的)測度はすべてσ‐有限なので,\cref{theorem:hopf_extension}から定まる拡張された測度は常に一意である.
\end{note}

\begin{proof}
証明のおおまかな流れだけ述べておく.各\(S\in\powerset{\Omega}\)に対し,\(\sigmaalg{A}\)上の集合列\(\seq{A_n}_{n\in\numset{N}}\)で\(S\subset\bigcup_{n\in\numset{N}}A_n\)を満たすもの全体を\(\operatornamewithlimits{cover}_{\sigmaalg{A}}S\)とおく.
そして,写像\(\mu^{\mathord{\ast}}\colon\powerset{\Omega}\to\ccival{0}{+\infty}\)を
\[
  \mu^{\mathord{\ast}}(S) = \inf\Set*{\sum_{n=1}^\infty m(A_n)\given\seq{A_n}_{n\in\numset{N}}\in\operatornamewithlimits{cover}_{\sigmaalg{A}}S}
\]
で定義する.すると,集合
\[
  \sigmaalg{F} = \Set{A\in\powerset{\Omega}\given\text{任意の\(E\in\powerset{\Omega}\)に対し\(\mu^{\mathord{\ast}}(A)=\mu^{\mathord{\ast}}(A\cap E)+\mu^{\mathord{\ast}}(A\setminus E)\)}}
\]
は完全加法族であり,\(\sigmaalg{A}\)を包含する.また,\(\mu^{\mathord{\ast}}\)の始域を\(\sigmaalg{F}\)へと制限した写像\(\bar{\mu}\)は,\(\sigmaalg{F}\)上の測度であることが示せる.
したがって,\(\bar{\mu}\)の\(\sigmagen{\sigmaalg{A}}\)への制限\(\mu\)は\(\sigmagen{\sigmaalg{A}}\)上の測度である.
\end{proof}

\(\symcal{E}\)を左半開区間\(\ocival{a}{b}\cap\numset{R}\)(\(-\infty\leq a<b\leq+\infty\))の全体集合とすると,集合\(\sigmaalg{A}=\Set{\emptyset}\cup\Set{\bigsqcup_{k=1}^nI_k\given\text{\(I_1,\dots,I_n\in\symcal{E}\)は互いに素}}\)は有限加法族をなす.
\(\sigmaalg{A}\)上の有限加法的測度\(\operatorname{vol}\)を
\[
  \operatorname{vol}\pqty*{\bigsqcup_{k=1}^n\ocival{a_k}{b_k}} = \sum_{k=1}^n(b_k-a_k)
\]
で定義する.このとき\(\operatorname{vol}\)はσ‐有限かつ\nameref{theorem:hopf_extension}の条件を満たすので,\(\sigmagen{\sigmaalg{A}}\)上の測度\(\mu\)へと一意に拡張できる.

実は\(\sigmagen{\sigmaalg{A}}=\borelset{\numset{R}}\)である.この\(\borelset{\numset{R}}\)上の測度\(\mu\)を\termdef{ボレル測度}\index{ぼれるそくど@ボレル測度}\index{そくど@測度!ぼれる@ボレル\twoemdash}(Borel measure)という.
また,\(m=\operatorname{vol}\)のときの\(\bar{\mu}\)を\termdef{ルベーグ測度}\index{るべーぐそくど@ルベーグ測度}\index{そくど@測度!るべーぐ@ルベーグ\twoemdash}(Lebesgue measure)という.

\section{ルベーグ積分}

\subsection{ルベーグ積分}

\begin{definition}{可測関数}{}\index{かそくかんすう@可測関数}
  \((\Omega,\sigmaalg{F})\)を可測空間とする.関数\(f\colon\Omega\to\extendedreal\)が\termdef{可測関数}(measurable function)であるとは,
  任意の\(A\in\borelset{\numset{R}}\)に対して\(\inv{f}{A}\in\sigmaalg{F}\)が成立することをいう.
\end{definition}

特に\((\Omega,\sigmaalg{F},\prob)\)が確率空間であるとき,可測関数のことを(実数値)\termdef{確率変数}\index{かくりつへんすう@確率変数}(random variable)ともいう.

\begin{proposition}{}{}
  集合\(S\subset\powerset{\numset{R}}\)が\(\sigmagen{S}=\borelset{\numset{R}}\)を満たすとき,任意の\(A\in S\)に対して\(\inv{f}{A}\in\sigmaalg{F}\)が成立すれば,\(f\)は可測関数である.
  特に,任意の\(a\in\numset{R}\)に対して\(\inv{f}{\ocival{-\infty}{a}}\in\sigmaalg{F}\)なら,\(f\)は可測関数である.
\end{proposition}

\begin{proof}
  集合\(\sigmaalg{G}=\Set{A\in\powerset{\numset{R}}\given\inv{f}{A}\in\sigmaalg{F}}\)はσ‐加法族であり,仮定から\(S\subset\sigmaalg{G}\)なので\(\sigmagen{S}\subset\sigmaalg{G}\)である.よって,\(f\)は可測関数である.
\end{proof}

\begin{definition}{単関数}{}\index{しじかんすう@指示関数}\index{たんかんすう@単関数}
  \((\Omega,\sigmaalg{F})\)を可測空間とする.\(a_1,\dots,a_n\in\numset{R}\)と互いに素な\(A_1,\dots,A_n\in\sigmaalg{F}\)を用いて
  \[
    \phi(x) = \sum_{k=1}^na_k\indicator_{A_k}(x)
  \]
  と表される関数\(\phi\colon\Omega\to\numset{R}\)を\termdef{単関数}(simple function)という.
  ただし\(\indicator_{A_k}\)は\termdef{指示関数}(indicator function)であり,\(x\in A_k\)のとき\(\indicator_{A_k}(x)=1\),\(x\notin A_k\)のとき\(\indicator_{A_k}(x)=0\)で定義される.
\end{definition}

可測関数\(f\)の積分\(\int f\measured{\mu}=\int f(x)\measured[x]{\mu}\)を定義しよう.まず,非負値単関数\(\phi=\sum_{k=1}^na_k\indicator_{A_k}\)の積分を
\[
  \int\phi\measured{\mu} = \sum_{k=1}^na_k\mu(A_k)
\]
で定義する.そして,非負値可測関数\(f\)の積分を
\[
  \int f\measured{\mu} = \sup\Set*{\int\phi\measured{\mu}\given\text{\(\phi\in\simplefuncs{\sigmaalg{F}}\),\(0\leq\phi\leq f\)}}
\]
で定義する.ただし,\(\simplefuncs{\sigmaalg{F}}\)は単関数の全体集合であり,\(\phi\leq f\)は任意の\(x\in\Omega\)に対し\(\phi(x)\leq f(x)\)であることを意味する.

\begin{wrapfigure}[8]{o}{0pt}
  \includegraphics{figures/integration.pdf}
\end{wrapfigure}

より具体的に,\(\int f\measured{\mu}\)を非負値単関数の積分に関する極限で表すこともできる.\(\midx{E}{n}{k}=\inv{f}{\coival{2^{-n}k}{2^{-n}(k+1)}}\)とする.このとき
\[
  \phi_n = n\indicator_{\inv{f}{\ccival{n}{+\infty}}}+\sum_{k=0}^{2^nn-1}\frac{k}{2^n}\indicator_{\midx{E}{n}{k}}
\]
は非負値単関数で,\(\int\phi_n\measured{\mu}\to\int f\measured{\mu}\)(\(n\to\infty\))である.

\(f\)が負の値をとりうる可測関数のときは,\(f^{\mathord{+}}(x)=\max\Set{f(x),0}\)と\(f^{\mathord{-}}(x)=\max\Set{-f(x),0}\)が非負値可測関数であることを利用して
\[
  \int f\measured{\mu} = \int f^{\mathord{+}}\measured{\mu}-\int f^{\mathord{-}}\measured{\mu}
\]
と定義する.ただし,\(\int f^{\mathord{-}}\measured{\mu}=\int f^{\mathord{+}}\measured{\mu}=+\infty\)のとき\(\int f\measured{\mu}\)は定義されない.
\(\int f^{\mathord{-}}\measured{\mu}\)と\(\int f^{\mathord{+}}\measured{\mu}\)がどちらも有限なとき,\(f\)は\termdef{可積分}\index{かせきぶん@可積分}(integrable)であるという.

また,集合\(S\in\sigmaalg{F}\)上での積分は
\[
  \int_Sf\measured{\mu} = \int_Sf(x)\measured[x]{\mu}
  = \int f(x)\indicator_S(x)\measured[x]{\mu}
\]
で定義される.

特に\((\Omega,\sigmaalg{F},\prob)\)が確率空間のとき,確率変数\(X\)の積分\(\int X(\omega)\measured[\omega]{\prob}\)を\(X\)の\termdef{期待値}\index{きたいち@期待値}\index{EX@\(\expval[X]\)}(expected value)といい,\(\expval[X]\)と書く.

\begin{example}
  \cref{example:equally_possible}の確率空間\((\Omega,\sigmaalg{F},\prob)\)において,各\(\omega\in\Omega\)に対し\(X(\omega)\)を\(\omega\)の目で定義する.すなわち\(X(\dicei)=1\),\(X(\diceii)=2\)のようにする.
  このとき\(X=1\indicator_{\Set{\dicei}}+2\indicator_{\Set{\diceii}}+\dots+6\indicator_{\Set{\dicevi}}\)だから
  \[
    \expval[X] = 1\prob({\Set{\dicei}})+2\prob(\Set{\diceii})+\dots+6\prob({\Set{\dicevi}})
    = \frac{1+2+\dots+6}{6}
    = \frac{7}{2}
  \]
  である.この場合,\(\expval[X]=7/2\)は公平な6面ダイスの出目の期待値に相当する.
\end{example}

\begin{proposition}{}{}
  実数値関数\(f\)は有界閉区間\(I=\ccival{a}{b}\)上で定義され有界とする.このとき,\(f\)がリーマン積分できれば\(f\)はルベーグ可測かつ可積分で
  \[
    \int_a^bf(x)\intd{x} = \int_If\measured{\lambda}\quad\text{(\(\lambda\)は\(\numset{R}\)上のルベーグ測度)}
  \]
  が成立する.
\end{proposition}

\subsection{収束定理}

\begin{theorem}{ファトゥの補題}{fatou}\index{ふぁとうのほだい@ファトゥの補題}
  \((\Omega,\sigmaalg{F},\mu)\)を測度空間とする.任意の可測関数列\(\seq{f_n}_{n\in\numset{N}}\)に対して
  \[
    \int\pqty*{\liminf_{n\to\infty}f_n(x)}\measured[x]{\mu} \leq \liminf_{n\to\infty}\int f_n(x)\measured[x]{\mu}
  \]
  である.これを\termdef{ファトゥの補題}(Fatou's lemma)という.
\end{theorem}

\begin{theorem}{単調収束定理}{monotone_convergence}\index{たんちょうしゅうそくていり@単調収束定理}\index{MCT|see{単調収束定理}}
  \((\Omega,\sigmaalg{F},\mu)\)を測度空間とする.可測関数列\(\seq{f_n}_{n\in\numset{N}}\)が\(0\leq f_1\leq f_2\leq\dotsb\)を満たすとき
  \[
    \lim_{n\to\infty}\int f_n(x)\measured[x]{\mu} = \int\pqty*{\lim_{n\to\infty}f_n(x)}\measured[x]{\mu}
  \]
  である.これを\termdef{単調収束定理}(monotone convergence theorem; MCT)という.
\end{theorem}

\begin{theorem}{優収束定理}{dominated_convergence}\index{ゆうしゅうそくていり@優収束定理}\index{DCT|see{優収束定理}}
  \((\Omega,\sigmaalg{F},\mu)\)は測度空間で,可測関数列\(\seq{f_n}_{n\in\numset{N}}\)は関数\(f\)に各点収束するとする.
  このとき,可測関数\(g\)ですべての\(n\in\numset{N}\)に対し\(\abs{f_n}\leq g\)を満たすものが存在すれば
  \[
    \lim_{n\to\infty}\int f_n\measured{\mu} = \int f\measured{\mu}
  \]
  である.これを\termdef{優収束定理}(dominated convergence theorem; DCT)という.
\end{theorem}

\end{document}
