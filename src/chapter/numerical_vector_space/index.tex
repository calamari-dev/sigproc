\documentclass[../../main]{subfiles}

\begin{document}
\chapter{数ベクトル空間}
\label{chapter:numerical_vector_space}

\begin{lead}
  \cref{chapter:numerical_vector_space}では,数ベクトル空間における直交性と最良近似の関係を説明する.
\end{lead}

\section{イントロダクション}
\label{section:numerical_vector_space_introduction}

\cref{chapter:numerical_vector_space}では,数ベクトル空間\(\numset{K}^n\)(\(\numset{K}=\numset{R},\numset{C}\))に関する理論を扱う.
信号解析において,この理論は
\begin{enumerate}
  \item 離散時間信号の時系列分析
  \item 観測値をモデルに対応づける回帰・判別分析
\end{enumerate}
という,2つの方向に応用される.

音声信号処理は前者の主要な例である.音声信号を計算機で処理するには,時々刻々と値が変わる信号を有限長のデータで表現しなければならない.
たとえば,CDでは音声信号の瞬時値を1秒あたり44100個記録している.すなわち,時刻\(t\)秒における瞬時値を\(x(t)\),収録時間を\(T\)秒とおくと,CDには数列\(\seq{x(n/44100)}_{n=0}^{44100T}\)が記録されている.
そこで,収録されたデータを\(\numset{R}^{44100T+1}\)の元とみなせば,\(\numset{R}^n\)に関する理論に基づいて音声を解析できる.

後者の主要な例は最小2乗法である.実験で得られた標本を理論と見比べるとき,理論から得られる式へのあてはめ(回帰)がしばしば試される.
あてはまりのよさを示す指標はいろいろあるが,最もポピュラーなのは2乗誤差を指標にする最小2乗法である.本書ではこの最小2乗法を,内積と関連づけ幾何的に説明する.

\section{直交射影}

本節では,あるベクトルを他のベクトルの線型結合で近似する手法を説明する.
特に断りのない限り,\cref{chapter:numerical_vector_space}において\(\numset{K}\)は\(\numset{R}\)か\(\numset{C}\)を意味し,
\(\innerp{\holder}{\holder}\)は\(\numset{K}^n\)の標準内積を意味する.また
\[
  \vnorm{\vect{x}} = \sqrt{\innerp{\vect{x}}{\vect{x}}}
  = \sqrt{\abs{x_1}^2+\dots+\abs{x_n}^2}
  \quad(\vect{x}=\trps{\matrice{x_1 & \cdots & x_n}}\in\numset{K}^n)
\]
とする\indexsymbol{\(\vnorm{\holder}\)}.

\subsection{直交射影}

\begin{wrapfigure}[8]{o}{0pt}
  \includegraphics{figures/proj2d.pdf}
\end{wrapfigure}

\(\numset{K}^n\)のベクトル\(\vect{x}\),部分空間\(V\)が与えられたとき,\(V\)の元で\(\vect{x}\)に最も近いベクトル,すなわち,距離\(\vnorm{\vect{x}-\vect{m}}\)を最小にする\(\vect{m}\in V\)について考えよう.

\(\numset{K}^n\)が平面\(\numset{R}^2\)で,\(V\)があるベクトル\(\vect{v}\neq\zvec\)により生成される直線\(\spannedby\Set{\vect{v}}\)の場合,\(\vect{m}\)は図の位置にある.
図を見ると,\(\vect{x}-\vect{m}\)は\(\vect{v}\)と直交しているのが分かる.

一般の部分空間\(V\subset\numset{K}^n\)においても,直交性と最良近似には密接な関係がある.その証明へと入る前に,便利な記法を2つ定義しておく.

\begin{definition}{argmin,argmax}{argmin_argmax}\index{argmin@\(\argmin f(x)\)}\index{argmax@\(\argmax f(x)\)}
  実数値関数\(f\)は集合\(S\)を定義域に含むとする.\(S\)の部分集合\(\argmin_{x\in S}f(x)\),\(\argmax_{x\in S}f(x)\)を以下の通り定義する.
  \begin{gather*}
    \argmin_{x\in S}f(x) = \Set{x\in S\given\text{任意の\(y\in S\)に対して\(f(y)\geq f(x)\)}}, \\
    \argmax_{x\in S}f(x) = \Set{x\in S\given\text{任意の\(y\in S\)に対して\(f(y)\leq f(x)\)}}
  \end{gather*}
\end{definition}

\cref{definition:argmin_argmax}からただちに,次のことが分かる.

\begin{proposition}{}{}
  \(S\)の元\(a\)に関する以下の条件は同値であり,同様のことが\(\argmax\)についても成り立つ.
  \begin{enumerate}
    \item \(a\in\argmin_{x\in S}f(x)\)である.
    \item \(f(a)\)は集合\(\ran{f}{S}=\Set{f(x)\given x\in S}\)の最小元である.
  \end{enumerate}
\end{proposition}

\begin{wrapfigure}[8]{o}{0pt}
  \includegraphics{figures/argmin.pdf}
\end{wrapfigure}

図は\(\napr^{-x}\)と\(\abs{\sin x}\)のグラフである.\(\napr^{-x}\to 0\)(\(x\to\infty\))であるが,\(\napr^{-x}=0\)となる実数\(x\)は存在しない.そのため
\begin{gather*}
  \argmin_{x\in\coival{0}{\infty}}\napr^{-x} = \emptyset\quad\text{(空集合)}, \\
  \argmin_{x\in\coival{0}{\infty}}\abs{\sin x} = \Set{0,\krez,2\krez,\dotsc}
\end{gather*}
である.このように,\(\argmin_{x\in S}f(x)\)は空になることも,無限集合になることもある.

\(\numset{K}=\numset{R}\)の場合も同様に証明できるので,\cref{proposition:finite_projection}まで証明では\(\numset{K}=\numset{C}\)を仮定する.また,部分空間が\(\Set{\zvec}\)でないことも仮定する.

\begin{lemma}{}{binomial_square}
  各\(\vect{x},\vect{y}\in\numset{K}^n\)に対して\(\vnorm{\vect{x}+\vect{y}}^2=\vnorm{\vect{x}}^2+2\rpart\innerp{\vect{x}}{\vect{y}}+\vnorm{\vect{y}}^2\)が成立する.
\end{lemma}

\begin{proof}
  \(\vnorm{\vect{x}+\vect{y}}^2=\innerp{\vect{x}+\vect{y}}{\vect{x}+\vect{y}}\)の右辺を展開すれば示せる.
\end{proof}

\begin{proposition}{}{finite_convex_projection}
  \(\vect{x}\in\numset{K}^n\)かつ,\(V\)は\(\numset{K}^n\)の部分空間とする.
  このとき,\(\argmin_{\vect{y}\in V}\vnorm{\vect{x}-\vect{y}}\)はただ1つの元からなる集合である.
\end{proposition}

\begin{proof}
  本証明に限り,\(\sum_{k=1}^m\)(\(m=\dim V\))を\(\sum\)と略記する.\(\basis{B}=\Set{\vect{e}_1,\dots,\vect{e}_m}\)を\(V\)の正規直交基底とする.
  このとき\(V=\Set{\sum z_k\vect{e}_k\given z_1,\dots,z_m\in\numset{C}}\)なので,\(\epsilon(z_1,\dots,z_m)=\vnorm{\vect{x}-\sum z_k\vect{e}_k}\)とおくと
  \[
    \argmin_{\vect{y}\in V}\vnorm{\vect{x}-\vect{y}} = \Set*{\sum z_k\vect{e}_k\given\trps{\matrice{z_1 & \cdots & z_m}}\in\argmin_{\vect{z}\in\numset{C}^m}\epsilon(\vect{z})}
  \]
  である.

  \(\argmin_{\vect{z}\in\numset{C}^m}\epsilon(\vect{z})\)を求める.\(\innerp{\vect{e}_i}{\vect{e}_j}=\kdelta{i}{j}\)だから
  \[
    \vnorm*{\sum_{i=1}^mz_i\vect{e}_i}^2 = \innerp*{\sum_{i=1}^mz_i\vect{e}_i}{\sum_{j=1}^mz_j\vect{e}_j}
    = \sum_{i=1}^mz_i\sum_{j=1}^m\conj{z}_j\innerp{\vect{e}_i}{\vect{e}_j}
    = \sum_{i=1}^mz_i\conj{z}_i
    = \sum_{i=1}^m\abs{z_i}^2
  \]
  である.したがって,\cref{lemma:binomial_square}より
  \begin{align*}
    \epsilon(\vect{z})^2 &= \vnorm*{\vect{x}-\sum z_k\vect{e}_k}^2 = \vnorm{\vect{x}}^2-2\rpart\innerp*{\vect{x}}{\sum z_k\vect{e}_k}+\vnorm*{\sum z_k\vect{e}_k}^2 \\
    &= \vnorm{\vect{x}}^2-2\sum\rpart[\conj{z}_k\innerp{\vect{x}}{\vect{e}_k}]+\sum\abs{z_k}^2
  \end{align*}
  である.よって,\(\epsilon(\vect{z})^2\)は\(s_k=\rpart z_k\)と\(t_k=\ipart z_k\)の式で
  \begin{align*}
    \epsilon(\vect{z})^2 &= \vnorm{\vect{x}}^2+\sum(-2\rpart[(s_k-\iuni t_k)\innerp{\vect{x}}{\vect{e}_k}]+s_k^2+t_k^2) \\
    &= \vnorm{\vect{x}}^2+\sum(-2(s_k\rpart\innerp{\vect{x}}{\vect{e}_k}+t_k\ipart\innerp{\vect{x}}{\vect{e}_k})+s_k^2+t_k^2) \\
    &= \vnorm{\vect{x}}^2+\sum((s_k-\rpart\innerp{\vect{x}}{\vect{e}_k})^2+(t_k-\ipart\innerp{\vect{x}}{\vect{e}_k})^2-\abs{\innerp{\vect{x}}{\vect{e}_k}}^2)
  \end{align*}
  と書けるので,次式が成立する.
  \begin{equation}
    \label{equation:pre_bessels_inequality}
    \epsilon(\vect{z})^2 = \vnorm{\vect{x}}^2+\sum_{k=1}^m\abs{z_k-\innerp{\vect{x}}{\vect{e}_k}}^2-\sum_{k=1}^m\abs{\innerp{\vect{x}}{\vect{e}_k}}^2
  \end{equation}

  \cref{equation:pre_bessels_inequality}より\(\argmin_{\vect{z}\in\numset{C}^m}\epsilon(\vect{z})=\Set{\trps{\rowvect{\innerp{\vect{x}}{\vect{e}_1} & \cdots & \innerp{\vect{x}}{\vect{e}_m}}}}\)であるから,
  \(\argmin_{\vect{y}\in V}\vnorm{\vect{x}-\vect{y}}=\Set{\sum\innerp{\vect{x}}{\vect{e}_k}\vect{e}_k}\)である.
\end{proof}

なお,\cref{proposition:finite_convex_projection}は部分空間よりも少し広い対象(閉凸集合)へと一般化できるのだが,そのことは\cref{xr-chapter:hilbert_space}であらためて扱う.

\begin{proposition}{}{weak_finite_projection}
  \(\vect{x}\in\numset{K}^n\)かつ,\(V\)は\(\numset{K}^n\)の部分空間とする.
  \(V\)のある元\(\vect{m}\)が任意の\(\vect{v}\in V\)に対して\(\innerp{\vect{x}-\vect{m}}{\vect{v}}=0\)を満たすとき,
  \(\vect{m}\in\argmin_{\vect{y}\in V}\vnorm{\vect{x}-\vect{y}}\)である.
\end{proposition}

\begin{proof}
  任意に\(\vect{y}\in V\)をとり,\(\vect{\epsilon}=\vect{y}-\vect{m}\)とおく.すると,\(\innerp{\vect{x}-\vect{m}}{\vect{\epsilon}}=0\)より
  \(\vnorm{\vect{x}-\vect{y}}^2=\vnorm{\vect{x}-\vect{m}-\vect{\epsilon}}^2=\vnorm{\vect{x}-\vect{m}}^2-2\rpart\innerp{\vect{x}-\vect{m}}{\vect{\epsilon}}+\vnorm{\vect{\epsilon}}^2=\vnorm{\vect{x}-\vect{m}}^2+\vnorm{\vect{\epsilon}}^2\)
  が成立する.よって\(\vnorm{\vect{x}-\vect{y}}\geq\vnorm{\vect{x}-\vect{m}}\)だから,\(\vect{m}\in\argmin_{\vect{y}\in V}\vnorm{\vect{x}-\vect{y}}\)である.
\end{proof}

\cref{proposition:weak_finite_projection}からは,仮定「任意の\(\vect{v}\in V\)に対して\(\innerp{\vect{x}-\vect{m}}{\vect{v}}=0\)」を満たす\(\vect{m}\in V\)が存在するかどうかは分からない.
しかし実は,仮定を満たす\(\vect{m}\)は一意に存在し,それは\(\argmin_{\vect{y}\in V}\vnorm{\vect{x}-\vect{y}}\)のただ1つの元である.

\begin{proposition}{}{finite_projection}
  \(\vect{x}\in\numset{K}^n\)かつ,\(V\)は\(\numset{K}^n\)の部分空間とする.
  このとき,\(V\)の元\(\vect{m}\)に関する以下の条件は同値であり,条件を満たす\(\vect{m}\)はただ1つ存在する.
  \begin{enumerate}
    \item \(\vect{m}\in\argmin_{\vect{y}\in V}\vnorm{\vect{x}-\vect{y}}\)である.
    \item 任意の\(\vect{v}\in V\)に対して\(\innerp{\vect{x}-\vect{m}}{\vect{v}}=0\)である.
  \end{enumerate}
\end{proposition}

\begin{proof}
  \cref{proposition:finite_convex_projection}より,\(\vect{n}\in\argmin_{\vect{y}\in V}\vnorm{\vect{x}-\vect{y}}\)を満たす\(\vect{n}\)がただ1つ存在する.
  そして\cref{proposition:weak_finite_projection}より,\(\vect{m}\in V\)が任意の\(\vect{v}\in V\)に対して\(\innerp{\vect{x}-\vect{m}}{\vect{v}}=0\)を満たすなら\(\vect{m}=\vect{n}\)である.

  したがって,\(\vect{n}\)がすべての\(\vect{v}\in V\)に対して\(\innerp{\vect{x}-\vect{n}}{\vect{v}}=0\)を満たすことを示せばよい.それには\(\vnorm{\vect{v}}=1\)のときについて示せば十分である.
  \(\vect{n}\)の定義から,関数\(d(z)=\vnorm{\vect{x}-(\vect{n}+z\vect{v})}^2-\vnorm{\vect{x}-\vect{n}}^2\)(\(z\in\numset{C}\))は負の値をとらない.一方,\(x=\rpart z\),\(y=\ipart z\)とおくと
  \begin{align*}
    d(z) &= \vnorm{(\vect{x}-\vect{n})-z\vect{v}}^2-\vnorm{\vect{x}-\vect{n}}^2
    = -2\rpart[\conj{z}\innerp{\vect{x}-\vect{n}}{\vect{v}}]+\abs{z}^2\vnorm{\vect{v}}^2 \\
    &= -2(x\rpart\innerp{\vect{x}-\vect{n}}{\vect{v}}+y\ipart\innerp{\vect{x}-\vect{n}}{\vect{v}})+x^2+y^2 \\
    &= (x-\rpart\innerp{\vect{x}-\vect{n}}{\vect{v}})^2+(y-\ipart\innerp{\vect{x}-\vect{n}}{\vect{v}})^2-\abs{\innerp{\vect{x}-\vect{n}}{\vect{v}}}^2 \\
    &= \abs{z-\innerp{\vect{x}-\vect{n}}{\vect{v}}}^2-\abs{\innerp{\vect{x}-\vect{n}}{\vect{v}}}^2
  \end{align*}
  なので\(\abs{\innerp{\vect{x}-\vect{n}}{\vect{v}}}^2=-d(\innerp{\vect{x}-\vect{n}}{\vect{v}})\leq 0\),よって\(\innerp{\vect{x}-\vect{n}}{\vect{v}}=0\)である.
\end{proof}

\begin{definition}{直交射影}{finite_projection}\index{ちょっこうしゃえい@直交射影}\index{proj@\(\proj_V(\vect{x})\)}
  \cref{proposition:finite_projection}の\(\vect{m}\)を\(\vect{x}\)の\(V\)への\termdef{直交射影}(orthogonal projection)といい,\(\proj_V(\vect{x})\)と表す.
\end{definition}

\begin{figure}[htbp]
  \centering
  \includegraphics{figures/proj3d.pdf}
  \caption{\(\vect{x}\)の\(V=\spannedby\Set{\vect{v}_1,\vect{v}_2}\)への直交射影\(\vect{m}=\proj_V(\vect{x})\)の模式図.}
\end{figure}

\begin{example}[\(xy\)平面への直交射影]
  \(\vect{e}_x=\trps{\rowvect{1 & 0 & 0}}\),\(\vect{e}_y=\trps{\rowvect{0 & 1 & 0}}\)とし,
  \(\numset{R}^3\)の部分空間\(V\)を\(V=\spannedby\Set{\vect{e}_x,\vect{e}_y}\)で定義する.
  このとき,集合\(\Set{\vect{e}_x,\vect{e}_y}\)は\(V\)の正規直交基底なので\(\proj_V(\vect{r})=\innerp{\vect{r}}{\vect{e}_x}\vect{e}_x+\innerp{\vect{r}}{\vect{e}_y}\vect{e}_y=\trps{\rowvect{x & y & 0}}\)(\(\vect{r}=\trps{\rowvect{x & y & z}}\in\numset{R}^3\))である.
\end{example}

\begin{proposition}{}{}
  \(\numset{K}^n\)の任意の部分空間\(V\)について,写像\(\proj_V\colon\numset{K}^n\to V\)は線型写像である.
\end{proposition}

\begin{proof}
  \(s,t\in\numset{K}\),\(\vect{x},\vect{y}\in\numset{K}^n\)を任意にとり,\(\vect{z}=s\vect{x}+t\vect{y}\),\(\vect{m}=s\proj_V(\vect{x})+t\proj_V(\vect{y})\)とおく.
  このとき,任意の\(\vect{v}\in V\)に対して\(\innerp{\vect{z}-\vect{m}}{\vect{v}}=s\innerp{\vect{x}-\proj_V(\vect{x})}{\vect{v}}+t\innerp{\vect{y}-\proj_V(\vect{y})}{\vect{v}}=s0+t0=0\)なので,
  \(\proj_V(\vect{z})=\vect{m}\)である.よって,\(\proj_V\)は線型写像である.
\end{proof}

\subsection{直交補空間}

\begin{definition}{直交補空間}{numerical_perpendicular_complement}\index{ちょっこうほくうかん@直交補空間}\indexsymbol{\(\pcomp{V}\)}\indexsymbol{\(\pcomp[V]{W}\)}
  \(V\)は\(\numset{K}^n\)の部分空間とする.\(W\)が\(V\)の部分空間なら,集合
  \[
    X = \Set{\vect{v}\in V\given\text{任意の\(\vect{w}\in W\)に対して\(\innerp{\vect{v}}{\vect{w}}=0\)}}
  \]
  も\(V\)の部分空間になる.\(X\)を(\(V\)における)\(W\)の\termdef{直交補空間}(orthogonal complement)といい,\(\pcomp[V]{W}\)と表記する.誤解のおそれがなければ,\(\pcomp[V]{W}\)を\(\pcomp{W}\)とも書く.
\end{definition}

\begin{example}
  \(W=\spannedby\Set{\vect{e}_1,\vect{e}_2}\)を\(\numset{R}^3\)の2次元部分空間とする.
  このとき,\(\numset{R}^3\)における\(W\)の直交補空間は,\(\vect{e}_1\)と\(\vect{e}_2\)に直交する\(\zvec\)でないベクトル\(\vect{e}_3\)で生成される直線\(\spannedby\Set{\vect{e}_3}\)である.
  特に\(\vect{e}_1\)と\(\vect{e}_2\)が直交するとき,集合\(\Set{\vect{e}_i/\vnorm{\vect{e}_i}\given i=1,2,3}\)は\(\numset{R}^3\)の正規直交基底である.
\end{example}

\begin{figure}[htbp]
  \centering
  \includegraphics{figures/orthogonal_complement.pdf}
  \caption{\(W\)と\(\vect{e}_1\),\(\vect{e}_2\),\(\vect{e}_3\)の様子.}
\end{figure}

\begin{proposition}{}{}
  \(V\)は\(\numset{K}^n\)の部分空間で,\(W\)は\(V\)の部分空間とする.このとき\(V=W\oplus\pcomp[V]{W}\)である.
\end{proposition}

\begin{proof}
  \(\vect{x}\in W\cap\pcomp{W}\)なら\(\innerp{\vect{x}}{\vect{x}}=0\)なので\(\vect{x}=\zvec\),よって\(W\cap\pcomp{W}=\Set{\zvec}\)である.
  また\cref{proposition:finite_projection}より,任意の\(\vect{x}\in V\)に対して\(\vect{x}-\proj_W(\vect{x})\in\pcomp{W}\),\(\vect{x}=\proj_W(\vect{x})+(\vect{x}-\proj_W(\vect{x}))\in W+\pcomp{W}\)である.
  したがって\(V=W\oplus\pcomp{W}\)である.
\end{proof}

\subsection{分析と合成}

\cref{proposition:finite_convex_projection}の証明では,\(\proj_V(\vect{x})\)の存在を示すために\(V\)の正規直交基底\(\basis{B}=\Set{\vect{e}_1,\dots,\vect{e}_m}\)を1つ選び,\(\proj_V(\vect{x})\)を\(\sum_{i=1}^m\innerp{\vect{x}}{\vect{e}_i}\vect{e}_i\)と表した.
一方で(特に信号解析では),\(\vect{x}\)の性質を調べるのに利用したい\(\numset{C}^n\)の正規直交基底\(\basis{B}=\Set{\vect{e}_1,\dots,\vect{e}_n}\)があって,
そこから部分空間\(V_m=\spannedby\Set{\vect{e}_1,\dots,\vect{e}_m}\)(\(m=1,\dots,n\))への直交射影\(\proj_{V_m}(\vect{x})\)を作ることも多い.そのような場合,直交射影は3つの操作に分解できる.

\begin{definition}{エルミート転置}{hermitian_transpose}\index{えるみーとてんち@エルミート転置}\index{ずいはんぎょうれつ@随伴行列|see{エルミート転置}}\index{H@\(\htrps{\mat{A}}\)}
  \(\mat{A}\)を\(m\times n\)複素行列とする.\(n\times m\)行列\(\trps{\conj{\mat{A}}}\)を\(\mat{A}\)の\termdef{エルミート転置}(Hermitian transpose)といい,\(\htrps{\mat{A}}\)と表す\footnotemark .
\end{definition}

\footnotetext{エルミート転置は\termdef{随伴行列}(adjoint matrix)と呼ばれることも多いが,別の行列を随伴行列と呼ぶ流儀もあり,まぎらわしい.そのため,本書ではエルミート転置で統一する.}

\(\mat{U}=\htrps{\rowvect{\vect{e}_1 & \cdots & \vect{e}_n}}\),\(\mat{\Lambda}=\begin{bsmallmatrix}\imat_m & \\ & \zmat_{n-m}\end{bsmallmatrix}\)とおく(\(\imat_m\)は\(m\)次単位行列,\(\zmat_{n-m}\)は\(n-m\)次零行列).
このとき,任意の\(\vect{x}=\trps{\rowvect{x_1 & \cdots & x_n}}\in\numset{C}^n\)に対して
\[
  \mat{U}\vect{x} = \matrice*{\htrps{\vect{e}_1}\vect{x} \\ \vdots \\ \htrps{\vect{e}_n}\vect{x}}
  = \matrice*{\innerp{\vect{x}}{\vect{e}_1} \\ \vdots \\ \innerp{\vect{x}}{\vect{e}_n}},
  \quad\mat{\Lambda}\vect{x} = \matrice*{x_1 \\ \vdots \\ x_m \\ \zvec},
  \quad\htrps{\mat{U}}\vect{x} = \htrps{\mat{U}}\matrice*{x_1 \\ \vdots \\ x_n}
  = \sum_{i=1}^nx_i\vect{e}_i
\]
であるから
\[
  \htrps{\mat{U}}\mat{\Lambda}\mat{U}\vect{x} = \htrps{\mat{U}}\mat{\Lambda}\matrice*{\innerp{\vect{x}}{\vect{e}_1} \\ \vdots \\ \innerp{\vect{x}}{\vect{e}_n}}
  = \htrps{\mat{U}}\matrice*{\innerp{\vect{x}}{\vect{e}_1} \\ \vdots \\ \innerp{\vect{x}}{\vect{e}_m} \\ \zvec}
  = \sum_{i=1}^m\innerp{\vect{x}}{\vect{e}_i}\vect{e}_i
  = \proj_{V_m}(\vect{x})
\]
であり,\(\proj_{V_m}(\vect{x})=\htrps{\mat{U}}\mat{\Lambda}\mat{U}\vect{x}\)が成立する.言い換えれば,\(\proj_{V_m}\)は\(\numset{C}^n\)から\(\numset{C}^n\)への3つの写像
\(T(\vect{x})=\mat{U}\vect{x}\),\(L(\vect{x})=\mat{\Lambda}\vect{x}\),\(\synth{T}(\vect{x})=\htrps{\mat{U}}\vect{x}\)を用いて,\(\proj_{V_m}=\synth{T}LT\)と表せる.

\(T(\vect{x})\)の第\(i\)成分\(\innerp{\vect{x}}{\vect{e}_i}\)は,\(\vect{x}\)に含まれる\(\vect{e}_i\)の「成分」を表すと考えられる.その理由は2つある.
1つめの理由は,\(\vnorm{\proj_{\spannedby\Set{\vect{e}_i}}(\vect{x})}=\vnorm{\innerp{\vect{x}}{\vect{e}_i}\vect{e}_i}=\abs{\innerp{\vect{x}}{\vect{e}_i}}\)なので,
\(\abs{\innerp{\vect{x}}{\vect{e}_i}}\)が\(\vect{e}_i\)のスカラー倍で\(\vect{x}\)を最もよく近似するベクトルの長さを表すことである.
もう1つの理由は,\(\basis{B}\)は\(\numset{K}^n\)の正規直交基底であるから
\begin{equation}
  \label{equation:analysis_and_synthesis}
  \vect{x} = \proj_{V_n}(\vect{x})
  = \sum_{i=1}^n\innerp{\vect{x}}{\vect{e}_i}\vect{e}_i
\end{equation}
が成立し,\(\innerp{\vect{x}}{\vect{e}_i}\vect{e}_i\)の和で\(\vect{x}\)が表されることである.

以上の理由から,本書では線型写像\(T(\vect{x})=\trps{\rowvect{\innerp{\vect{x}}{\vect{e}_1} & \cdots & \innerp{\vect{x}}{\vect{e}_n}}}\)を分析作用素,\(\synth{T}(\vect{x})=\sum_{i=1}^nx_i\vect{e}_i\)を合成作用素と呼ぶ.

\begin{definition}{分析作用素,合成作用素}{analysis_and_synthesis}\index{ぶんせきさようそ@分析作用素}\index{ごうせいさようそ@合成作用素}
  \(\basis{B}=\Set{\vect{e}_1,\dots,\vect{e}_n}\)を\(\numset{K}^n\)の正規直交基底とする.
  \begin{enumerate}
    \item 線型写像\(T\colon\numset{K}^n\to\numset{K}^n\),\(T(\vect{x})=\trps{\rowvect{\innerp{\vect{x}}{\vect{e}_1} & \cdots & \innerp{\vect{x}}{\vect{e}_n}}}\)を\(\basis{B}\)に関する\termdef{分析作用素}(analysis operator)という.
    \item 線型写像\(\synth{T}\colon\numset{K}^n\to\numset{K}^n\),\(\synth{T}(\trps{\rowvect{x_1 & \cdots & x_n}})=\sum_{i=1}^nx_i\vect{e}_i\)を\(\basis{B}\)に関する\termdef{合成作用素}(synthesis operator)という.
  \end{enumerate}
\end{definition}

\cref{equation:analysis_and_synthesis}より,合成作用素は分析作用素の逆写像である.
また,分析作用素と合成作用素が持つ性質は,表現行列に関する条件へと言い換えられる.

\begin{definition}{正規行列,ユニタリ行列}{regular_and_unitary_matrix}\index{せいきぎょうれつ@正規行列}\index{ゆにたりぎょうれつ@ユニタリ行列}
  \(\mat{A}\)を\(n\)次複素正方行列とする.
  \begin{enumerate}
    \item \(\htrps{\mat{A}}\mat{A}=\mat{A}\htrps{\mat{A}}\)であるとき,\(\mat{A}\)を\termdef{正規行列}(normal matrix)という.
    \item \(\htrps{\mat{A}}\mat{A}=\mat{A}\htrps{\mat{A}}=\imat\)であるとき(つまり\(\htrps{\mat{A}}=\mat{A}^{-1}\)であるとき),\(\mat{A}\)を\termdef{ユニタリ行列}(unitary matrix)という.
  \end{enumerate}
\end{definition}

\cref{definition:regular_and_unitary_matrix}から,ユニタリ行列は正規行列である.また,次の命題が成立する.

\begin{proposition}{ユニタリ行列の性質}{}
  \(\mat{U}=\rowvect{\vect{u}_1 & \cdots & \vect{u}_n}\)を\(n\)次複素正方行列とする.このとき,以下の命題は同値である.
  \begin{enumerate}
    \item \(\mat{U}\)はユニタリ行列である.
    \item 集合\(\Set{\vect{u}_1,\dots,\vect{u}_n}\)は\(\numset{C}^n\)の正規直交基底である.
  \end{enumerate}
\end{proposition}

\begin{proof}
  \(\htrps{\mat{U}}\mat{U}=[\midx{a}{i}{j}]\)とおくと
  \[
    \htrps{\mat{U}}\mat{U} = \matrice*{\htrps{\vect{u}_1} \\ \vdots \\ \htrps{\vect{u}_n}}\matrice{\vect{u}_1 & \cdots & \vect{u}_n}
    = \matrice*{\htrps{\vect{u}_1}\vect{u}_1 & \cdots & \htrps{\vect{u}_1}\vect{u}_n \\ \vdots & \ddots & \vdots \\ \htrps{\vect{u}_n}\vect{u}_1 & \cdots & \htrps{\vect{u}_n}\vect{u}_n}
  \]
  なので\(\midx{a}{i}{j}=\htrps{\vect{u}_i}\vect{u}_j=\innerp{\vect{u}_j}{\vect{u}_i}\)である.よって,\(\mat{U}^{-1}=\htrps{\mat{U}}\)であることと,各\(i,j\in\Set{1,\dots,n}\)に対して\(\innerp{\vect{u}_i}{\vect{u}_j}=\kdelta{i}{j}\)であることは同値である.
\end{proof}

\begin{corollary}{}{}
  \(T\colon\numset{C}^n\to\numset{C}^n\)を線型写像とする.このとき,以下の命題は同値である.
  \begin{enumerate}
    \item \(T\)はある正規直交基底に関する分析作用素(合成作用素)である.
    \item 標準基底に関する\(T\)の表現行列はユニタリ行列である.
  \end{enumerate}
\end{corollary}

\section{離散フーリエ変換}

本節から,本書の主題である信号解析に入っていく.

\subsection{離散フーリエ変換}

\begin{definition}{離散フーリエ変換}{discrete_fourier_transform}\index{りさんふーりえへんかん@離散フーリエ変換}\index{DFT|see{離散フーリエ変換}}\index{FZn@\(\dft{n}\vect{x}\)}
  各\(\vect{x}=\trps{\rowvect{x_0 & \cdots & x_{N-1}}}\in\numset{C}^N\)に対して,\(\numset{C}^N\)の元
  \[
    \hat{\vect{x}} = \trps{\matrice{\hat{x}_0 & \cdots & \hat{x}_{N-1}}},
    \quad\hat{x}_k = \frac{1}{\sqrt{N}}\sum_{n=0}^{N-1}x_n\napr^{-2\krez\iuni kn/N}
  \]
  を対応づける線型写像\(\dft{N}\colon\numset{C}^N\to\numset{C}^N\)を\termdef{離散フーリエ変換}(Discrete Fourier transform; DFT)という.
\end{definition}

以下では\(\napr^{2\krez\iuni/N}=\cos(2\krez/N)+\iuni\sin(2\krez/N)\)を\(\zeta_N\),もしくは単に\(\zeta\)と書く.

\begin{proposition}{}{dft_onb}
  \(\vect{w}_k=(1/\sqrt{N})\trps{\rowvect{\zeta^{k\cdot 0} & \cdots & \zeta^{k(N-1)}}}\)とする.
  このとき,集合\(\Set{\vect{w}_0,\dots,\vect{w}_{N-1}}\)は\(\numset{C}^N\)の正規直交基底である.
\end{proposition}

\begin{proof}
  \(\conj{\zeta}=\zeta^{-1}\)だから,\(\innerp{\vect{w}_i}{\vect{w}_j}=\trps{\vect{w}_i}\conj{\vect{w}}_j\)は
  \[
    \sum_{n=0}^{N-1}\frac{\zeta^{in}}{\sqrt{N}}\frac{\conj{\zeta}^{jn}}{\sqrt{N}} = \frac{1}{N}\sum_{n=0}^{N-1}\zeta^{(i-j)n}
    = \begin{cases}(\zeta^{(i-j)N}-1)/(N(\zeta^{i-j}-1)) & (i\neq j), \\ 1 & (i=j)\end{cases}
  \]
  と変形できる.\(\zeta^N=1\)なので\(\innerp{\vect{w}_i}{\vect{w}_j}=\kdelta{i}{j}\)である.
\end{proof}

\cref{proposition:dft_onb}から,\(\dft{N}\)は正規直交基底\(\basis{W}=\Set{\vect{w}_0,\dots,\vect{w}_{N-1}}\)に関する分析作用素である.
分析作用素の逆写像は合成作用素なので,\(\dft{N}\)の逆変換は
\[
  \vect{x} = \sum_{k=0}^{N-1}\hat{x}_k\vect{w}_k,
  \quad x_n = \frac{1}{\sqrt{N}}\sum_{k=0}^{N-1}\hat{x}_k\napr^{2\krez\iuni kn/N}
\]
と書ける.

\begin{proposition}{}{dft_plancherel}
  \(\hat{\vect{x}}=\dft{N}\vect{x}\),\(\hat{\vect{y}}=\dft{N}\vect{y}\)とすると,\(\innerp{\hat{\vect{x}}}{\hat{\vect{y}}}=\innerp{\vect{x}}{\vect{y}}\)が成立する.
\end{proposition}

\begin{proof}
  標準基底に関する\(\dft{N}\)の表現行列を\(\mat{W}\)とおく.このとき
  \(\innerp{\hat{\vect{x}}}{\hat{\vect{y}}}=\innerp{\mat{W}\vect{x}}{\hat{\vect{y}}}=\trps{\vect{x}}\trps{\mat{W}}\conj{\hat{\vect{y}}}=\trps{\vect{x}}\conj*{\htrps{\mat{W}}\hat{\vect{y}}}=\innerp{\vect{x}}{\htrps{\mat{W}}\hat{\vect{y}}}\)
  であり,\(\mat{W}\)はユニタリ行列なので\(\htrps{\mat{W}}\hat{\vect{y}}=\htrps{\mat{W}}\mat{W}\vect{y}=\vect{y}\),\(\innerp{\vect{x}}{\htrps{\mat{W}}\hat{\vect{y}}}=\innerp{\vect{x}}{\vect{y}}\)となる.
\end{proof}

\cref{proposition:dft_plancherel}に関して,特に\(\vect{x}=\vect{y}\)のとき
\begin{equation}
  \label{equation:dft_parseval}
  \vnorm{\dft{N}\vect{x}}^2 = \vnorm{\vect{x}}^2,
  \quad\sum_{k=0}^{N-1}\abs{\hat{x}_k}^2 = \sum_{n=0}^{N-1}\abs{x_n}^2
\end{equation}
である.信号処理ではしばしば\cref{equation:dft_parseval}を\termdef{パーセヴァルの定理}\index{ぱーせばるのていり@パーセヴァルの定理}(Parseval's theorem),あるいは\termdef{プランシュレルの定理}\index{ぷらんしゅれるのていり@プランシュレルの定理}(Plancherel's theorem)と呼ぶ.

さらなる諸性質を導く前に,離散フーリエ変換の工学的重要性を見ておこう.
\cref{figure:time_domain}は「あ」という音声の波形である\footnote{出典は波音リツ単独音 Ver1.5.1 \cite{canon}.}.

\begin{figure}[htbp]
  \centering
  \includegraphics{figures/time_domain.pdf}
  \caption{「あ」の波形.}
  \label{figure:time_domain}
\end{figure}

\cref{section:numerical_vector_space_introduction}に述べた要領で,\cref{figure:time_domain}のデータを数ベクトル\(\vect{x}\in\numset{R}^N\)と見なそう.
\(\vect{x}\)を離散フーリエ変換すると,\(\abs{\hat{x}_k}\)(\(k=0,\dots,N-1\))は\cref{figure:frequency_domain}のようになる.

\begin{figure}[htbp]
  \centering
  \includegraphics{figures/frequency_domain.pdf}
  \caption{「あ」のスペクトル.}
  \label{figure:frequency_domain}
\end{figure}

\cref{figure:frequency_domain}を見ると,\SI{250}{Hz}周辺にピークが現れている.

\subsection{エイリアシング}

\begin{figure}[htbp]
  \centering
  \includegraphics{figures/aliasing.pdf}
  \caption{エイリアシングの様子.}
\end{figure}

\subsection{多次元離散フーリエ変換}

\begin{definition}{多次元離散フーリエ変換}{multidimensional_discrete_fourier_transform}\index{りさんふーりえへんかん@離散フーリエ変換!たじげん@多次元}\index{FZdn@\(\dft[d]{\vect{n}}x\)}
  \(N_1,\dots,N_d\)を自然数の組とし,\(\Omega=\Set{\trps{\rowvect{u_1 & \cdots & u_d}}\given\text{\(u_j\in\numset{Z}\),\(0\leq u_j<N_j\)(\(1\leq j\leq d\))}}\),\(\mat{N}=\diag(N_1,\dots,N_d)\)とおく.
  関数\(x\colon\Omega\to\numset{C}\)に対して,関数
  \[
    \hat{x}(\vect{k}) = \frac{1}{\sqrt{\det\mat{N}}}\sum_{\vect{n}\in\Omega}x(\vect{n})\napr^{-2\krez\iuni\trps{\vect{k}}\mat{N}^{-1}\vect{n}}\quad(\vect{k}\in\Omega)
  \]
  を対応づける線型写像\(\dft[d]{\mat{N}}\colon\numset{C}^\Omega\to\numset{C}^\Omega\)を\termdef{\(d\)次元離散フーリエ変換}という.
\end{definition}

特に\(d=2\)のとき
\begin{align*}
  \hat{x}(k_1,k_2) &= \frac{1}{\sqrt{N_1N_2}}\sum_{n_2=0}^{N_2-1}\sum_{n_1=0}^{N_1-1}x(n_1,n_2)\napr^{-2\krez\iuni(k_1n_1/N_1+k_2n_2/N_2)} \\
  &= \frac{1}{\sqrt{N_2}}\sum_{n_2=0}^{N_2-1}\pqty*{\frac{1}{\sqrt{N_1}}\sum_{n_1=0}^{N_1-1}x(n_1,n_2)\napr^{-2\krez\iuni k_1n_1/N_1}}\napr^{-2\krez\iuni k_2n_2/N_2}
\end{align*}
であり,右辺は\(x(n_1,n_2)\)を各変数に関して離散フーリエ変換した形になっている.
より一般に,\(x(n_1,\dots,n_d)\)の\(d\)次元離散フーリエ変換は,\(x(n_1,\dots,n_d)\)を各変数に関して離散フーリエ変換したものと一致する.

\section{最小2乗問題}

本節では,直交射影の理論を近似へと応用する.

\subsection{最小2乗問題}

\subsection{スペクトル定理}

\subsection{特異値分解}

\subsection{擬似逆行列}

\section{多重解像度解析}

\section{主成分分析}

\begin{figure}[htbp]
  \begin{minipage}{\linewidth/2}
    \centering
    \includegraphics{figures/scatter.pdf}
  \end{minipage}%
  \begin{minipage}{\linewidth/2}
    \centering
    \includegraphics{figures/pca.pdf}
  \end{minipage}
\end{figure}

\begin{subappendices}
\section{低ランク近似}
\end{subappendices}

\section*{演習問題}
\addcontentsline{toc}{section}{演習問題}

\end{document}
