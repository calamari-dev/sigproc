\documentclass[../../main]{subfiles}

\begin{document}
\chapter{数ベクトル空間}
\label{chapter:numerical_vector_space}

\begin{lead}
  \cref{chapter:numerical_vector_space}で書く予定のことを並べておく.
\end{lead}

\section{行列と線型空間}
\subsection{固有値と固有空間}
\begin{definition}{固有値,固有空間}{eigenvalue}\index{こゆうち@固有値}\index{こゆうべくとる@固有ベクトル}
\(\mat{A}\)を\(n\)次正方行列とする.
複素数\(\lambda\)と\(\zvec\)でないベクトル\(\vect{x}\in\numset{C}^n\)が式\(\mat{A}\vect{x}=\lambda\vect{x}\)を満たすとき,
\(\lambda\)を\(\mat{A}\)の\termdef{固有値}(eigenvalue)という.
また,\(\vect{x}\)を\(\mat{A}\)の(固有値\(\lambda\)に属する)\termdef{固有ベクトル}(eigenvector)という.
\end{definition}

\begin{definition}{固有空間}{eigenspace}\index{こゆうくうかん@固有空間}
\cref{definition:eigenvalue}の\(\mat{A}\),\(\lambda\)について,集合
\[
  E_\lambda(\mat{A}) = \Set{\vect{x}\in\numset{C}^n\given\mat{A}\vect{x}=\lambda\vect{x}}
\]
は\(\numset{C}^n\)の部分空間になる.部分空間\(E_\lambda(\mat{A})\)を,
\(\mat{A}\)の(固有値\(\lambda\)に属する)\termdef{固有空間}(eigenspace)という.
\end{definition}

\subsection{対角化}

\section{直交射影}
\subsection{直交射影}
\subsection{直交補空間}
\subsection{スペクトル定理}

\section{最小二乗問題}
\subsection{最小二乗問題}
\subsection{特異値分解}
\subsection{擬似逆行列}

\section{離散フーリエ変換}

\section{多重解像度解析}

\begin{subappendices}
\section{主成分分析}
\section{低ランク近似}
\section{窓関数}
\end{subappendices}

\section*{演習問題}
\addcontentsline{toc}{section}{演習問題}

\end{document}
